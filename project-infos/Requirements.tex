\documentclass{article}

\usepackage[utf8]{inputenc}  % To handle UTF-8 encoding
\usepackage{amsmath}         % For mathematical symbols and formulas

\title{Data Visualization Final Project}
\author{Diego Chiola , Gabriele Berruti , Alex Valle}
\date{\today}                % Automatically inserts today's date

\begin{document}

\maketitle                   % Generates the title, author, and date

\section{Requirements}

\subsection{- github repo (invite annalisabarla as member)}
Github Link: https://github.com/Berrutz/DataVis-Project

\subsection{- github pages URL}
Github Page: https://berrutz.github.io/DataVis-Project/

\subsection{- self-evalutation of web programming skills}
The three team members' programming abilities are comparable to those developed during the first three years of the Bachelor's degree in 
Computer Science at the University of Genoa.
This covers knowledge of HTML, CSS, and JavaScript, as well as hands-on experience with current frameworks like React and Angular.

\subsection{- technologies that you plan to use for web programming (e.g. JavaSript frameworks)}
All of the technologies employed are listed here:

\begin{itemize}

    \item \textbf{React} \\
    That's because React is a JavaScript library for creating user interfaces, specifically for single-page apps. 
    It enables developers to reuse UI components, effectively manage state, and dynamically update the DOM using a virtual DOM system.

    \item \textbf{Tailwind} \\
    Tailwind CSS is a utility-first CSS framework that includes low-level, pre-defined classes for creating bespoke designs directly in HTML. 
    It allows for quick styling without creating unique CSS, making it very adaptable and efficient for current web development.

    \item \textbf{Next.js} \\
    Next.js is a popular React framework used for building web applications.
    It includes various features to improve pages load speed like Server-Side Rendering (SSR) and Static Site Generation (SSG).
    Node.js also lets you create API endpoints within your application. This means you can handle both front-end and back-end
    logic in one project without needing a separate back-end server.

    \item \textbf{Yarn} \\
    Yarn is dependency management tool for JavaScript projects.
    It is known for surpassing dependency management tool like npm in terms of speed, security and reliability.

    \item \textbf{D3.js} \\
    D3.js is a JavaScript toolkit that lets you create dynamic, interactive data visualizations in web browsers. 
    It employs HTML, SVG, and CSS to bring data to life, giving you fine-grained control over the visual components.

\end{itemize}


\subsection{- strategies/technologies you are going to use for data handling/processing}
Regarding the technologies to manage data pre-processing, we opted for:

\begin{itemize}

    \item \textbf{Pandas} \\
    Pandas is a Python package designed for data manipulation and analysis. 
    It offers data structures like as DataFrames for effectively handling and transforming huge datasets.

\end{itemize}

\end{document}
